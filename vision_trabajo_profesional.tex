\documentclass[a4paper,11pt]{article}
\usepackage[spanish] {babel}
\title { Visi\'on Trabajo Profesional }
\date{2do cuatrimestre de 2009}
\author{Ciancio Alessio, Mauro Lucas \\ Gilioli, Leandro Ezequiel }

\begin{document}
	\maketitle

	\section{Objetivo}
		El objetivo de este documento es analizar y definir los requerimientos y
		necesidades de alto nivel para el proyecto a presentarse como Trabajo Profesional.

	\section{Alcance}
	    El proyecto consiste en un sistema de edici\'on de documentos de texto plano
	    multiusuario que permita a varios usuarios confeccionar un documento en forma
	    colaborativa. \\
	    Los usuarios editar\'an en tiempo real un mismo documento que es compartido por uno
	    de ellos o centralizado en un servidor. Los cambios que introduce un usuario 
	    ser\'an visibles instantaneamente por los dem\'as.	
	    
		El nombre del proyecto es \textit{Parallel Editor}. \\

		Los usuarios de este proyecto son aquellas personas que trabajen produciendo 
		documentos de c\'odigo fuente o texto plano en general, como por ej. desarrolladores
		de software, dise\~nadores de p\'aginas web, estudiantes realizando trabajos pr\'acticos,
		etc.

		Cada d\'ia es m\'as com\'un la necesidad de trabajar en conjunto, estar en contacto
		con colegas de forma remota y desarrollar en equipo en forma distribuida. Este proyecto
		les puede aportar un ahorro de tiempo considerable al poder ver, editar y mejorar un
		documento por varios usuarios a la vez.

	    Hoy en d\'ia los entornos de desarrollo impuestos como est\'andar de facto proveen escaso
	    soporte para este tipo de soluciones. Se pretende lograr una soluci\'on que no obligue al
	    usuario a cambiar las herramientas que actualmente maneja, sino proveer una integraci\'on
	    con las mismas a fin de disminuir la curva de aprendizaje.

	\section{Posicionamiento}
		El problema surge de la necesidad de comunicaci\'on y retroalimentaci\'on constante.
		Los usuarios necesitan que su trabajo sea revisado por sus colegas de forma de asegurar
		calidad y est\'andares. Para este tipo de trabajo es indispensable que la herramienta sea
		eficiente y los documentos se compartan instantaneamente y puedan mejorarse colaborativamente.

		\subsection{Usuarios}
		    Como se nombr\'o anteriormente, existen diversos tipos de usuarios en este proyecto:

		    \begin{itemize}
				\item 	Los principales son los desarrolladores de cualquier lenguaje. Les permitir\'a
						compartir con sus pares las soluciones a problemas, requerir ayuda en
						determinados temas, aumentar la productividad obteniendo un \textit{feedback} mas
						rapidamente, etc.

				\item 	Desarrolladores de p\'aginas web: debido a que la mayor\'ia de los formatos
						de las p\'aginas web son formatos de texto.

				\item	Desarrollo de documentos de texto: documentos basados en texto o en Latex.

				\item	Estudiantes: para el desarrollo de trabajos pr\'acticos.

		    \end{itemize}

		\subsection{Interesados}
		    A continuaci\'on se revisar\'a la lista de interesados:

		    \begin{itemize}
				\item 	Compa\~nias de IT cuya principal actividad es el desarrollo de software.

				\item 	Equipos de trabajo con alto nivel de concurrencia sobre un mismo documento.

				\item 	Equipos de trabajo geogr\'aficamente distribuidos.

				\item	Entornos de desarrollo integrados (IDE) con arquitectura de soporte de plugins.
		    \end{itemize}

		    El problema impacta fuertemente en ambientes o situaciones d\'onde se requiere
		    la elaboraci\'on de un documento de forma colaborativa y en tiempo real.

		    Una soluci\'on a este problema es la que se pretende mostrar en este documento.

	\section{Descripci\'on general del Producto}
		La soluci\'on permitir\'a a usuarios que esten conectado entre s\'i a trav\'es de
		una LAN o Internet compartir un documento de texto e ir aportando contenido al mismo en
		tiempo real. En el escenario m\'as t\'ipico un usuario denominado \textit{host} o
		anfitri\'on crear\'a un servicio en la red para que los restantes puedan ingresar y
		comenzar el desarrollo.

		Una vez establecida la sesi\'on de desarrollo todos los usuarios pueden aportar
		informaci\'on al documento en tiempo real y al mismo tiempo observar los cambios
		introducidos por los dem\'as. La soluci\'on tendr\'a soporte para evitar problemas de
		concurrencia y para mantener la consistencia del documento.

	    En cualquier momento usuarios pueden ingresar o dejar la sesi\'on de desarrollo. Los
	    nuevos usuarios obtendr\'an la versi\'on actual del documento que se encuentran editando.

	    Varios usuarios podr\'an colaborar en varios documentos simultaneamente, no estando
	    limitado solo a trabajar en un documento a la vez.

		\subsection{M\'odulos}
			Basicamente se realizar\'an tres m\'odulos principales:

				\subsubsection{Nucleo}
					El primer m\'odulo ser\'a el nucleo o \textit{kernel} de la soluci\'on.
					Este m\'odulo ser\'a el encargado de comunicarse con los integrantes de
					la sesi\'on, enviar los cambios ingresados, crear y borrar documentos,
					crear y borrar sesiones de desarrollo, etc.

				\subsubsection{Interfaz Gr\'afica}
					El segundo m\'odulo ser\'a una interfaz gr\'afica que le permitir\'a al
					usuario realizar las operaciones desde un alto nivel. Opciones para crear
					sesiones de desarrollo, invitar usuarios a la sesi\'on, sacar usuarios de
					la sesi\'on, posibilidad de compartir mensajes privados entre los usuarios,
					abrir y cerrar documentos, ver las revisiones del documento que se est\'a
					editando, etc.

				\subsubsection{Integraci\'on con IDE}
					El tercer m\'odulo realizar\'a la integraci\'on con un entorno de desarrollo
					preexistente. Existen excelentes herramientas para realizar diversos
					proyectos en Java, C, C++, Python, etc; ya que no se busca reinventar la rueda,
					este m\'odulo se integrar\'a con esas herramientas de forma tal de aprovechar sus
					beneficios. Por integraci\'on se entiende editar un archivo concurrentemente en
					el mismo IDE aprovechando todas sus funcionalidades (completaci\'on de c\'odigo,
					\textit{refactor}, etc.).
\end{document}