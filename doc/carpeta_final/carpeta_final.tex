\documentclass[12pt,a4paper]{article}
\usepackage[utf8x]{inputenc}
\usepackage[spanish] {babel}
\usepackage{makeidx}
\usepackage[pdftex]{graphicx}


\title { \textbf{Trabajo Profesional}}
\date{2do cuatrimestre de 2010}
\author{\textbf{Ciancio Alessio, Mauro Lucas} \\
		\textbf{Gilioli, Leandro Ezequiel}	  \\
		\texttt{\{maurociancio,legilioli\}@gmail.com}
	}
% faltan padrones en algun lado %

\begin{document}
\maketitle
\tableofcontents
\newpage

	\section{Motivación}
Hoy en día el desarrollo de software en sus diversas tareas ha dejado de ser un trabajo 
puramente individual sino que requiere interacción de varias personas que en conjunto 
suman sus capacidades para lograr un producto de calidad superior.

Bajo esta forma de trabajo es necesario disponer de herramientas que ayuden a que los tiempos
requeridos por la coordinación e interacción de los integrantes de un grupo de trabajo sean bajos
de forma tal que el grupo sea productivo.

Con este objetivo, las herramientas utilizadas deben estar diseñadas e implementadas apuntando a
integrarse con las metodologías y herramientas existentes sin representar un obstáculo para el usuario
y a un bajo costo.

La idea nace de la necesidad de participar de una sesión de desarrollo, en la cual los participantes
no están físicamente en el mismo lugar. En algunos casos en particular, esta tarea requiere un
\textit{feedback} instantáneo entre los participantes que con herramientas ya existentes no se puede 
ofrecer.

Las herramientas existentes no logran que el feedback sea lo suficientemente rápido dado que trabajan
en un ambiente en el cuál es necesario respetar un protocolo para la modificación del estado de un 
documento. Por ejemplo: un usuario que está trabajando sobre un determinado archivo debe realizar todos
los cambios que quiere introducir, guardar la nueva versión del archivo modificado y enviarlo a sus
colaboradores para que estos esten al tanto de los cambios que introdujo (feedback). Mientras dura
este proceso los colaboradores no pueden realizar cambios al documento y si lo hacen será necesario
que realicen un proceso comunmente llamado \textit{merge}, de forma tal que el estado final del documento 
sea el mismo para cada uno de los participantes. Este proceso es lento y se hace especialmente engorroso
al aumentar el número de colaboradores.

Este esquema funciona bien en los casos en los cuales es baja la concurrencia sobre los mismos documentos,
es decir la edición de un mismo documento por parte de mas de un usuario es ocasional o en períodos de
tiempo disjuntos. Ejemplo: varios desarrolladores trabajando en un mismo proyecto con código fuente
compartido. Para estos casos existen herramientas cuya efectividad está comprobada, sistemas de control
de versiones (SVN, GIT, Mercurial) o servidores de archivos compartidos.

La solución desarrollada en este trabajo profesional resuelve los siguientes problemas:

\begin{itemize}
	\item Distribución geográfica: no es necesario estar en la misma ubicación física para que el proceso
	de desarrollo sea eficiente.
	\item Sin necesidad de un proceso de merge: el proceso de merge es realizado por el software en cada
	sitio de edición garantizando que el estado final del documento es el mismo para todos los 
	participantes. De esta manera se ahorra tiempo y se reducen los errores frecuentes o retrabajos
	los cuales son derivados de estos procesos.
	\item Alta latencia del feedback: los cambios en el estado del documento son reflejados en tiempo 
	real para todos los participantes.
\end{itemize}

La solución hace uso de tecnologías y herramientas existentes para proporcionar las funcionalidades 
que resuelven los problemas antes descriptos. Como ejemplo de esto, la solución se integró dentro de
Eclipse, el entorno de desarrollo integrado de facto para el lenguaje de programación Java.

	\section{Otras soluciones similares}
Durante la etapa de concepción del proyecto se analizaron otras soluciones similares al problema
anteriormente explicado. La descripción de cada una de ellas junto con la comparación de las mismas
respecto la presente solución fue detallada en el documento Propuesta de Trabajo Profesional. % cita aca %

A modo de resumen se presenta el cuadro comparativo \ref{soluciones_comparacion}:

\begin{table}[ht]
    \begin{tabular}{ | p{2.5cm} | p{5cm} | p{5cm} | }
    \hline
    Solución & Características & Comparativa \\ \hline

    Google Docs & Edicion de documentos en tiempo real desde un navegador. &
    Sólo puede utilizarse a través de un navegador e Internet. Código fuente cerrado. \\ \hline

    Google Wave & Comunicación y colaboración en tiempo real. &
    Idem Google Docs. El proyecto ha sido abandonado por Google. \\ \hline

    COLA (ECF) & Integración con Eclipse para colaboración en tiempo real de código fuente. &
	Limita a dos usuarios la cantidad de participantes en una sesión. Depende del proyecto ECF. \\ \hline

    BeWeeVee & Framework para integración de funcionalidades de colaboración en tiempo real para
    la plataforma .NET. & Código fuente cerrado. Está desarrollado sólo para la plataforma .NET. \\ \hline

    \end{tabular}
    \caption{\label{soluciones_comparacion} Tabla comparativa de soluciones}
\end{table}

	\section{Justificación teórica de los algoritmos}
	
	El problema que resuelve este trabajo profesional se presenta genéricamente en escenarios en los cuales 
	existen dos o mas partes que interactuan sobre un modelo (ya sea un documento o cualquier otro tipo de 
	objeto que almacene estado) aplicando operaciones sobre el mismo.
	
	Cada operación modifica el estado del modelo que es común a todas las partes. La cantidad de tipos distintos
	de operaciones que se pueden aplicar sobre el modelo depende de la naturaleza del mismo. Por ejemplo, en un
	modelo que representa un documento de texto las operaciones pueden ser inserción o borrado de texto, en el caso
	de un modelo que represente un area de dibujo se podrán aplicar operaciones de dibujado de figuras, 
	borrado, coloreado, etcétera.
	
	La existencia de varios participantes compartiendo un modelo para su edición (sucesivas aplicaciones de
	operaciones) hace que sea posible que más de un partipante puede estar realizando operaciones sobre el modelo.
	Al cabo de la aplicación de cada una de ellas todos las partes deben ver el mismo estado final, es decir,
	debe ser consistente. Por esta razón hay que tener en cuenta una manera de que esta condicion se alcance.

	El enfoque tradicional propone utilizar un modelo de sincronización centralizado en el cual para aplicar una
	operación sobre el modelo es necesario que el participante obtenga antes un control o autorización de un
	arbitro central. Una vez aplicada la operación, el estado resultante se transmite a todas las demás ubicaciones
	para que estén al tanto de los cambios generados como consecuencia de la aplicación de la operación remota.
	Si bien este método logra su cometido al preservar la consistencia del documento en todas las ubicaciones, la
	interactividad percibida en cada una de ellas es poco satisfactoria. Si al momento de querer aplicar una 
	operación no se tiene la autorización o control del modelo, se deberá esperar para obtenerlo. 
	Este tiempo de espera dependerá de la cantidad de participantes que se encuentran operando sobre el modelo
	en ese instante y del tiempo que cada operación remota tome en aplicarse (ejemplos de uso de este mecanismo
	son sistemas tradicionales de control versiones con bloqueo de archivos).

	Para evitar este inconveniente y dar la impresión de una aplicación instantánea de las operaciones en cada
	una de las ubicaciones de los participantes, el \textit{modelo de la transformada operacional} (OT) puede
	aplicarse. En éste, cada participante tiene una copia del modelo, sobre la cual aplica instantaneamente
	las operaciones que genera localmente. Luego de esto, notifica a todos los demás participantes que forman
	parte del proceso colaborativo.
	Al momento en que se reciben operaciones remotas en una ubicación, estas no son aplicadas, sino que se analiza
	el tipo de operación que se recibió y la secuencia de operaciones que fueron aplicadas en esa ubicación
	anteriormente. A partir de esta información es posible transformar la operación original en otra que cumple
	la propiedad de que al aplicarse al modelo local garantiza que el estado resultante será el mismo que el de
	los modelos de todas las demas ubicaciones.

	El algoritmo de sincronización que se implemento fue el algoritmo de \textbf{Jupiter} \cite{jupiter}. 
	El proceso de implementación de este algoritmo se divide en varias etapas.
	
	Como primer punto es necesario definir que tipo de componentes se va a utilizar para representar el modelo
	de documento. En esta solución se trabajará exclusivamente con el componente denominado cuadro de texto
	(\textit{textfield} en ingles). El cuadro de texto es un componente gráfico en el cual se puede insertar texto en 
	posiciones definidas por el usuario y existe en la mayoría de las bibliotecas de interfaces gráficas 
	disponibles actualmente. Es posible considerar otros tipos de componentes que no se implementan en esta
	solución, como por ejemplo: una zona para dibujar entra varios usuarios, un slider, botones, checkboxes, etc.
	
	Una vez que se definieron los componentes es necesario definir las operaciones que podrán ser aplicadas sobre
	los mismos. Estas pueden ser definidas como las posibles modificaciones que se hacen sobre el componente.
	Por ejemplo, para el caso de cuadro de texto una posible operación es agregar el texto \textsf{"hola"} 
	en la posición 10.
	Las operaciones necesarias para modificar el estado de un documento de texto plano en el componente utilizado
	en este trabajo son	dos: agregar texto en una posición y borrar texto en una posición.
	Es deseable mantener la cantidad de operaciones distintas y la cantidad de parámetros de las mismas al
	mínimo por razones que se explicarán mas adelante.

	Por último, una vez que se definieron los componentes a utilizar y las operaciones a realizar sobre los mismos,
	la implementación del algoritmo de sincronización que fue utilizado en esta solución requiere de una función 
	de transformación de operaciones que cumple con ciertas características y se denomina
	\textit{xform } \cite{jupiter}. 

	Para comprender la importancia de la utilización de un algoritmo de sincronización y la transformación
	de operaciones realizadas por la función \textit{xform }se procederá a plantear el siguiente escenario:
	
	Suponemos que dos usuarios están compartiendo un documento colaborativamente. El estado del documento 
	inicial en ambas ubicaciones es el mismo y corresponde a la palabra \texttt{"HOLA"}.
	El primer caso que se plantea como ejemplo corresponde a modificaciones al documento de manera tal 
	que las mismas no se superponen en el tiempo.
	En la ubicación 1 se inserta el caracter \texttt{'S'} en la posición 4. Esta operación se aplica localmente
	y es transmitida a la ubicación número 2. En este instante el resultando la copia local del documento es
	la palabra \texttt{"HOLAS"}.
	Mientras tanto en la ubicación 2 que posee la copia original del documento (\texttt{"HOLA"}), cuando se recibe
	la operación generada por la ubicación 1 se aplica localmente resultando ambas copias del documento idénticas.
	De igual forma en la ubicación 2 se genera otra operación (\textit{borrar caracter H}) que se aplica localmente
	y luego es transmitida a la ubicación 1. En esta se aplica la operación recibida y ambas ubicaciones terminan 
	con el mismo estado del documento.
	
	Esta interacción en el tiempo puede verse en la figura \ref{secuencia_ops_1}:

	\begin{figure}[!ht]
		\begin{center}
			%\includegraphics[width=14cm]{diagramafeliz.png}
			\caption{\label{secuencia_ops_1} dos ubicaciones aplican operaciones sobre el modelo }
		\end{center}
	\end{figure}

	En el caso presentado, no se observan problemas de sincronización debido a que la segunda operación fue 
	generada cuando la primera ya había sido procesada en la ubicación 2.

	Ahora, se presenta otro ejemplo en el que sí existe un problema de sincronización. Las partes luego de procesar
	las operaciones tendrán documentos distintos. Esto sucede dado que la ubicación 2 genera una operación local 
	sin aún haber recibido la operación generada por la otra parte.
	
	Nuevamente, el estado inicial del documento en ambas ubicaciones es \texttt{HOLA}. En la ubicación 1
	se generará la operación \textit{insertar C en posición 0} llevando el documento al estado \texttt{CHOLA}.
	Antes de que se reciba la operación en la ubicación 2, el usuario en esta ubicación generará la operación
	de borrado del caracter O. La operación es transmitida como \textit{"borrar caracter en posición 1"} ya que
	su estado actual es \texttt{HOLA}. Luego de aplicarse localmente la operación el estado del documento en
	el sitio 2 es \texttt{HLA}.
	En este momento ambas operaciones se encuentran viajando hacia las otras partes. En la ubicación 1 cuando recibe
	la operación generada remotamente, se aplica llevando al documento en el estado \texttt{COLA} ya que la
	operación indicaba borrar un caracter en la posición 1. Luego, en la ubicación 2 es recibida la operación
	generada en el sitio 1. Al aplicarse localmente lleva el documento a un estado \texttt{CHLA}.
	Como se observa, en ambos sitios se observa un documento distinto.

	Este caso se observa en el siguiente diagrama de la figura \ref{secuencia_ops_2}.
	
	\begin{figure}[!ht]
		\begin{center}
			%\includegraphics[width=14cm]{diagramatriste.png}
			\caption{\label{secuencia_ops_2} dos ubicaciones aplican operaciones sobre el modelo }
		\end{center}
	\end{figure}


	La divergencia en el estado final del documento se produce por que se aplicaron las operaciones en cada 
	ubicación sin haber sido transformadas previamente. En el caso de operaciones no conflictivas la función
	transformación de operaciones será la función  identidad. Sin embargo, para los casos en los cuales las
	operaciones son conflictivas la operación resultante transformada será distinta a la original.
	Para este último caso la operación 2 recibida desde la ubicación 2 debió haber sido corregida por la función
	de transformación y haber resultado en \textit{“borrar caracter en posición 2”} (y no 1 como originalmente
	era la operación).
	
	Presentada esta problemática es necesario definir una función \textit{xform} con las siguientes propiedades:
	
	\begin{equation} xform(op1,op2) = \lbrace op1’,op2’ \rbrace
	\end{equation}

	dónde \textit{op1} es la operacion generada por el usuario en la ubicación 1 y \textit{op2} es la operación
	generada por el usuario en la ubicación 2. La aplicación de la función da como resultado otro par de operaciones
	\textit{op1’} y \textit{op2’} que cumplen con la propiedad de que si la ubicación 1 aplica \textit{op1} seguida
	de  \textit{op2’}  y si la ubicación 2 aplica \textit{op2} seguida de \textit{op1’}, entonces ambas
	ubicaciones terminarán con el mismo estado del documento.
	Esto requiere que \textit{op1} y \textit{op2} hayan sido generadas a partir del mismo estado del documento.

	Para el caso anterior, se mostrará a continuación la función \textit{xform}.
	Dado el estado inicial del documento \texttt{HOLA}.

	En la ubicación 1 se aplica \textit{op1} y al momento en el que se recibe \textit{op2} se la transforma 
	para obtener la operación que debe aplicarse. A su vez, en la ubicación 2 se procede de forma similar
	aplicando primero localmente \textit{op2} y transformando \textit{op1} cuando se recibe.

	En ambas ubicaciones las operaciones transformadas se obtienen de la aplicación de \textit{xform}:
	
\begin{eqnarray*}
  xform(Insertar\ Caracter\ C\ en\ Pos\ =\ 0, \\
  Borrar\ Caracter\ en\ Pos\ =\ 1) & = & \\ 
  \left\lbrace Insertar\ Caracter\ C\ en\ Pos\ =\ 0, \\
  Borrar\ Caracter\ en\ Pos\ =\ 2\ \rbrace  
\end{eqnarray*}
% poner label a esto %

	De aquí resulta que \textit{op1’} es \textit{Insertar Caracter C en Pos = 0} y \textit{op2’} es 
	\textit{Borrar Caracter en Pos = 2}.

\textit{op1’} resultó ser igual a \textit{op1} mientras que \textit{op2’} fue desplazada con respecto a 
\textit{op2.}

En la ubicación 1 se aplica \textit{op2’} sobre el \texttt{CHOLA} resultando en: \texttt{CHLA}.

En la ubicación 2 se aplica \textit{op1’} sobre \texttt{HLA} resultando en \texttt{CHLA}.

Nótese que aplicando las operaciones transformadas se logra la convergencia del estado del documento 
en las dos ubicaciones.

\subsection{Definición \textit{xform}}

La definición de esta función puede resultar compleja si se toma en cuenta la cantidad de 
combinaciones de las operaciones aplicables al modelo. Por esta razón se debe mantener al mínimo la 
cantidad de operaciones para simplificar el desarrollo de la función.

La función de transformación debe contemplar los casos en las que las operaciones sean conflictivas entre 
sí. Dos operaciones \textit{op1} y \textit{op2} son conflictivas si el estado final del documento al aplicar 
primero \textit{op1} y luego \textit{op2} es distinto al que se obtiene primero \textit{op2} y luego \textit{op1}.
En el caso en que las operaciones no son conflictivas la transformación de las mismas es la transformación 
identidad.

La implementación de la función \textit{xform} en la presente solución puede observarse en la clase 
\texttt{BasicXFormStrategy}. La misma fue basada en los papers de Júpiter e INRIA. % cita aquí %

	\section{La búsqueda de la solución}

El paper Jupiter fue la base para la implementación del algoritmo de sincronización. Este paper fue 
encontrado durante el desarrollo de la visión del producto. Otros papers están basados en este y
corrigen y proponen nuevas soluciones o correcciones.

Se detectaron problemas al utilizar operaciones de longitud mayor a uno (enfoque original que se le
dio a la implementación). En un primer momento se detectaron problemas al utilizar operaciones de borrado
de más de un caracter. Luego surgieron problemas al usar inserciones de más de un caracter. Para solucionar
este problema se convirtieron las operaciones a operaciones de un caracter.

Este cambio no posee efectos colaterales importantes. Lo que se debe tener en cuenta es convertir las
operaciones generadas por la aplicación cliente (o API gráfica) a operaciones de un caracter.

Durante el desarrollo de las pruebas se encontró un caso denominado Puzzle que producia una falla en la
sincronización. Se denomina Puzzle a una combinación de estados y operaciones producidas en determinadas
ubicaciones que provoca que el algoritmo no garantize la convergencia del estado del documento para todos 
los usuarios. El paper de INRIA propone una solución a este problema que demuestra formalmente las 
propiedades del algoritmo de sincronización.
%cita en INRIA %

Se implementaron los cambios propuestos por este paper junto con los tests para probar el correcto funcionamiento.


\subsection{Problemas enfrentados durante el desarrollo}

\begin{itemize}
	\item El lenguaje de programacion Scala es una tecnología relativamente nueva (nacida en 2003) que si bien es
	estable cuenta con herramientas poco maduras para su desarrollo. Principalmente, el IDE no es tan completo
	funcionalmente como lo es al desarrollar en el lenguaje Java.

	\item El desarrollo se comenzó utilizando el IDE Eclipse pero la funcionalidad proveida era muy limitada 
	por lo que se cambió al Intellij IDEA. Este resultó ser uno de los IDEs con mayor soporte para el desarrollo
	en Scala.

	\item \textit{Scheduling} de actores: para procesos que tienen que ver con I/O se utilizaron \textit{threads}
	para no bloquear los actores restantes. Se encontró una biblioteca llamada \textit{Akka Actors} que provee una mejor 
	funcionalidad de actores, pero se decidió no cambiar a la misma ya que el proyecto estaba arrancado y 
	evitar demoras en el calendario. Provee una mejor facilidad para la configuración del scheduling de los
	actores.

    \item Se encontró una herramienta denominda SBT (\textit{Simple Build Tool}) que agiliza la compilación de proyectos 
    desarrollados en scala, pero la transición de Maven a SBT no resultaba trivial por lo que se descartó. El 
    beneficio de usar esta herramienta es que permite incrementar la velocidad de compilación con respecto a 
    usar el IDE o Maven.
\end{itemize}

\section{Riesgos}

Los riesgos detallados en la propuesta de trabajo profesional son los siguientes:

\subsection{ID 1: Demoras en el desarrollo de las funcionalidades por la poca experiencia con Scala}
Este riesgo no se materializó, en gran parte ya que se hizo una investigación y capacitación del lenguaje de
programación previamente al inicio del desarrollo. Los resultados obtenidos demuestran que fue lo suficiente
para afrontar el proyecto. La comunidad de programadores está en constante crecimiento y es muy activa y ayudó
durante el desarrollo con los problemas que surgieron.

\subsection{ID 2: Demoras en el desarrollo de la integración con Eclipse debido al desconocimiento de la 
arquitectura del mismo}
Se encontró que la documentación disponible de la arquitectura de Eclipse es abundante y de buena calidad por
lo cual no hubo mayores inconvenientes en la etapa de desarrollo del \textit{plugin}. Este riesgo no se se
materializó.

\subsection{ID 3: Demoras en el proyecto por la no aprobación de la propuesta de trabajo profesional}
La aprobación de la propuesta se produjo según lo planeado en el calendario, aproximadamente un mes después del
inicio del proyecto. No fue necesario cambiar el alcance del proyecto.

\subsection{ID 4: Demoras en la definición de la arquitectura produce retrabajo}
Se trabajó intensamente en una definición temprana de la arquitectura que resultó estable para el proyecto. Las
decisiones arquitecturales fueron tomadas en conjunto con todo el equipo de desarrollo y esto produjo un buen
concenso y entendimiento de la misma.

\subsection{Riesgos no relevados}
Durante el desarrollo se materializó un riesgo no contemplado originalmente. La estimación de algunas tareas dentro
de los sprints realizados estuvo muy desviado del valor real de horas hombre invertidas. Principalmente el desvío se
dio en las tareas de implementación del algoritmo de sincronización, se estimó un valor mucho menor al real (ver
\textit{Sprint Backlog 2}). El desvío fue compensado por la materialización de otro riesgo que fue la sobreestimación
de tareas dejando con un margen para poder cerrar el \textit{Sprint} con toda la funcionalidad convenida.
Los tres sprint realizados fueron cerrados con las tareas de mayor prioridad completas. En algunos casos se pasaron 
tareas hacia siguientes \textit{Sprint}, pero estas eran de baja prioridad.

\section{Elección de las tecnologías utilizadas}

\subsection{Lenguaje de programación Scala}
En retrospectiva la elección de este lenguaje para el desarrollo de los principales modulos de la solucion fue
acertada. El principal beneficio que se obtuvo fue una gran simplificación al utilizar un modelo de concurrencia basado
en actores. Este modelo abstrae al desarrollador de la complejidad de utilizar mecanismos de sincronización y comunicación
entre hilos como se realiza tradicionalmente en java.

Por otro lado, es un lenguaje multiparadigma (Orientado a Objectos y Funcional). La ventaja de esto es que permite utilizar
construcciones de lenguajes formales que resultan más concisas para resolver determinadas problematicas.

El lenguaje pone el foco en la inmutabilidad de los objetos haciendolo ideal para trabajar en ambientes concurrentes.
La sintaxis del lenguaje Scala resultó altamente expresiva y concisa. La proporción de código fuente entre Scala y Java es
la siguiente:

% diagrama de git hub %

Teniendo en cuenta que el plugin para el IDE Eclipse fue lo único que se desarrolló en el lenguaje Java puede
notarse que comparado con este, el código Scala es altamente denso, es decir que permite lograr gran fucionalidad con pocas
lineas de código.

Los módulos Cliente, Kernel, GUI, Common y Server fueron totalmente desarrollados en Scala mientras que el Plugin para
Eclipse fue implementado en Java. La compatibilidad es total entre estos dos lenguajes en los dos sentidos. Este punto
fue clave para el desarrollo del plugin y para la elección de Scala como lenguaje base. La misma resulto de gran utilidad
y aceleró el proceso de desarrollo.

\subsection{Eclipse}
Es el entorno integrado de desarrollo de facto para el lenguaje de programación Java y otros. La arquitectura está basada en
plugins y está concebida para que la extensión de funcionalidad sea realizada por medio de ellos. Posee una gran cantidad
de plugins desarrollados y una sólida documentación de la API.

Lo que se logró en el desarrollo del plugin del presente trabajo es que el mismo pueda integrarse con el IDE y agregarle
las funcionalidades sin interferir con plugins existentes. Por ejemplo, el desarrollo es compatible con todas las funcionalidades
incorporadas a los editores de texto. Por ejemplo: coloreo de código fuente, refactor, identado, formateo, generación de código
fuente.

Por otro lado, Eclipse provee un proceso de despliegue e instalación de plugins sencillo. Estos son publicados en Internet en
cualquier servidor web y la instalación de los mismos se realiza apuntando a la URL del mismo.

\subsection{Maven}
Es una herramienta que se utiliza para controlar de ciclo de vida del proyecto en términos de compilación, testing, resolución
de dependencias y despliegue. Provee integración con Scala.

\subsection{Spring}
Se utilizó Spring para la inyección de dependencias tanto el módulo GUI como en el módulo Server. La integración con Scala
no presentó inconvenientes.

\subsection{GIT}
Se utilizó para el control de versiones junto con GitHub. Es una herramienta útil ya que en varias oportunidades se
trabajó de manera offline.

\subsection{JUnit/EasyMock/TestNG}
Frameworks utilizados para el testing tanto en Scala como en Java.

\section{Arquitectura de la solución}

\subsection{Módulos}

Como primer punto se mostrarán los componentes que conforman la solución. Está compuesto por 3 módulos principales:

\begin{itemize}
	\item Kernel: provee la funcionalidad para inicializar servicios de colaboración de documentos.
	\item Cliente: define la API mediante la cual un cliente se conecta a un servidor colaborativo y edita documentos
	en tiempo real. Este es el módulo que utilizan las aplicaciones de terceros para incorporar las funcionalidades de
	edición colaborativa. Este módulo es utilizado tanto por el módulo GUI como por el módulo Eclipse.
	\item Common: incluye tipos comunes que utilizan los primeros dos módulos, como ser los mensajes del protocolo que 
	utilizan cliente y servidor para comunicarse y la estrategia de sincronización.
\end{itemize}


Además existen los siguientes módulos construídos en base a los anteriores:
\begin{itemize}
	\item Server: es una aplicación que crea un servicio para compartir documentos que se quieren editar colaborativamente
	a través de una red TCP/IP.
	\item GUI: es una aplicación cliente implementada usando Swing que ofrece la funcionalidad para conectarse a un servicio
	remoto y editar documentos existentes. Ofrece la posibilidad de compartir y editar documentos en un servidor colaborativo.
	\item Eclipse Plugin: integra la funcionalidades del módulo cliente y kernel dentro del IDE eclipse. Hace uso tanto del
	módulo cliente (para conectarse a servidor de colaboración existentes) como del módulo kernel (para la creación de 
	servidores de colaboración locales).
\end{itemize}

Las dependencias entre los módulos puede ser observada en el siguiente diagrama.

% diagrama de componentes %

\subsection{Despliegue de la solución}
Existen dos esquemas de despliegue para la solución.

\subsubsection{Cliente-Servidor}
El primer caso consiste en disponer de un servidor dedicado para la gestión de documentos compartidos. El mismo recibe
peticiones de N clientes que mediante una suscripción a los documentos pueden comenzar a realizar operaciones sobre los mismos.

El servidor es el encargado de reflejar los cambios introducidos por un cliente en los restantes.

Un diagrama de esta configuración puede observarse a continuación:

% diagrama de despliegue cliente servidor %

Los procesos involucrados en este tipo de despliegue son:

\begin{itemize}
	\item Proceso servidor: en este proceso se encuentra ejecutandose los siguientes módulos: Kernel, Common y Server. Sólo
	hay una instancia de este proceso.
	\item Proceso cliente: en este proceso se encuentra ejecutandose los siguientes modulos: Cliente, Common y la aplicación
	correspondiente que puede ser GUI o Eclipse. La cantidad de instancias de este proceso dependen de la cantidad de usuarios que
	haya conectados al servidor de colaboración.
\end{itemize}

\subsubsection{Peer-To-Peer}
La segunda configuración para el despliegue de la solución consiste en no disponer de un servidor dedicado, sino que el
mismo es ejecutado por uno de lo pares. Este enfoque es el que se utiliza en el plugin, ya que el mismo permite crear un servicio
a partir de un archivo que se quiere compartir.

% [[ diagrama de despliegue peer-2-peer]]

Esta configuración es la más sencilla para un uso ocasional del servicio o cuando la participantes en la sesión de edición no
es elevada. Las ventajas de esta configuración es que no se necesita un equipo central.

Si los participantes se encuetran en la misma LAN, no se requiere ninguna configuración extra. Si la sesión quiere realizarse
a través de internet es necesario tener en cuenta que haya conectividad entre las partes (configuración de puertos y firewall).

Al momento de escribir esta documentación no es posible utilizar el servicio detrás de un proxy. En cambio si posible 
hacerlo utilizarlo usando una VPN.

Los procesos involucrados en esta configuración son:
\begin{itemize}
	\item Proceso cliente con servicio de colaboración: Uno de los participantes involucrados será en encargado de crear el
	servicio de colaboración. En este se ejecutarán los siguientes módulos: Cliente, Kernel, Common y la aplicación que
	corresponda: GUI o Plugin Eclipse. Un sólo proceso existirá de este tipo.
	\item Procesos cliente sin servicio de colaboración: los restantes participantes se comunicarán con el cliente que posee
	el servicio de colaboración. Los módulos que se ejecutarán en este tipo de cliente son: Cliente y Common. La cantidad de
	estos procesos dependerá de la cantidad de usuarios conectados al servicio de colaboración.
\end{itemize}

\subsection{Despliegue de actores en proceso kernel}

En el siguiente diagrama se observan los actores desplegados en el kernel.

% [[ diagrama de actores en modulo kernel ]]

Los actores fueron utilizados para cumplir el rol de intermediario ante el acceso a un recurso, serializando los accesos de
modo de que no se produzcan conflictos de naturaleza concurrente.

Los principales actores del módulo kernel son los siguientes:
\begin{itemize}
	\item Kernel Actor: actúa como intermediario del objeto Kernel. La clase kernel gestiona las sesiones de los usuarios
	conectados y los documentos disponibles.
	\item Document Actor: actúa como intermediario de un objeto Document aplicando operaciones de los clientes sobre el mismo
	y replicándolas hacia los clientes restantes. Mantiene el estado del documento de modo que pueda servirlo a nuevas sesiones.
	\item Client Actor: es la representación de un cliente remoto frente al Kernel. Se encarga de recibir mensajes del
	cliente remoto, enviarlos al kernel y enviar la respuesta al mismo. Los mensajes que se reciben de la red son capturados por
	el actor NetworkListener mientras que los mensajes que se deben al cliente remoto son enviados por el GatewayActor.
\end{itemize}

Existe un protocolo interno representado por los mensajes que se envían entre sí. Los mensajes que conforman este protocolo
pueden verse en los siguientes archivos.
% archivo 1 COMPLETAR

\subsection{Despliegue de actores en el módulo cliente}
En el siguiente diagrama se observan los actores que se ejecutan en las aplicaciones que incorporar el módulo cliente.

% [[ diagrama de actores del módulo cliente ]]

Como se observa, el esquema es similar al encontrado en el Kernel. Existe un Actor denominado ClientActor que es el
encargado de coordinar los mensajes que deben enviarse hacia el kernel y de recibir las respuestas del mismo.
Los actores NetworkListenerActor y GatewayActor son los encargados de leer y escribir mensajes por la red, respectivamente.
En el módulo cliente se define una interfaz llamada Documents que es la que debe implementar la aplicación cliente para poder
utilizar los servicios provistos por la API.
A través de esta interfaz los mensajes son propagados hacia la aplicación cliente. Es responsabilidad que la aplicación cliente
responda a estos mensajes.

Los mensajes que son enviados a la aplicación cliente pueden encontrarse en el siguiente archivo:
* ar/noxit/paralleleditor/client/ClientMessages.scala

Con respecto a la cardinalidad de estos actores, podemos notar lo siguiente:
Por cada conexión que se realiza a un servidor de colaboración es creada una instancia de cada uno de estos actores.

Cabe aclarar que estos actores se encuentran detras de un facade, para evitar que la aplicación cliente observe la
complejidad subyacente. Este facade se denomina Session cuando es utilizado desde Scala y JSession cuando es utilizado desde Java.
Se hizo esta diferencia para que los nombres de métodos sean legibles desde Java. (Nombres de métodos que son ilegales en Java son
legales en Scala y al ser compilados a bytecode son convertidos resultando en nombres no convencionales. Ejemplo: el nombre de 
método ! es convertido a \$bang).


\subsection{Protocolo utilizado para enviar mensajes por la red}
Los mensajes son serializados e hidratados utilizando el mecanismo de serializacion de objetos de Java. Los mensajes que se
envian por la red se encuentran en el siguiente archivo:
ar/noxit/paralleleditor/common/messages/RemoteMessages.scala

Las clases que necesitan ser serializadas se encuetran marcadas con la anotación @serializable.

Se tuvo en cuenta durante la etapa de diseño que el protocolo sea facilmente reemplazable por otro. A continuación listamos
algunas posibles razones por la cual se podría reemplazar el protocolo por otro:

\begin{itemize}
	\item Seguridad: agregar una capa que implemente cifrado de datos.
	\item Compresión: compresión de los mensajes para reducir la carga de la red.
\end{itemize}

En el archivo ar/noxit/paralleleditor/common/network/SocketNetworkConnection.scala se implementa el protocolo actual. Referirse
al mismo para analizar posibles cambios.

\subsection{Diagrama de clases del kernel}
A continuación se muestra un diagrama de las clases principales que trabajan en el Kernel.

% [[ diagrama de clases en el kernel]]

Por simplicidad no se incluyen métodos de determinadas clases.

En el mismo se muestra que interfaces son utilizadas por los clientes. Los clientes son los usuarios que se conectan
remotamente al servicio de colaboración. Por lo tanto, el actor Cliente mostrado en el diagrama es un actor remoto y se
ocupa el rol de intermediario entre el Kernel y el Cliente real que se encuentra del otro lado de la conexión.

Las interfaces que utiliza son las siguientes:
\begin{itemize}
	\item Kernel: utiliza esta interfaz para pedir servicios al kernel de login, servicios de suscripción a documentos,
	servicios de chat, etc.

	\item Session: esta interfaz representa una sesión con el kernel. Todas las operaciones sobre el kernel requieren que se
	esté logueado al mismo. Esta interfaz ofrece la posibilidad de instalar una callback por la cual será notificada de mensajes
	provenientes del kernel. Entre estos mensajes podemos encontrar: mensajes de chat de otro usuario, mensajes de actualización
	de estado de un documento, etc. Es responsabilidad del cliente de instalar una callback para poder escuchar estos mensajes.

	\item Document Session: representa la sesión de edición de un documento en particular. Mediante esta interfaz es posible
	comunicarse con el documento para poder enviarle modificaciones.
\end{itemize}

\subsection{Como integrar esta solución a productos de terceros}
Para completar la documentación sobre la arquitectura de la solución se explicará como integrar este producto en otros ya
existentes. Se mostrará que no es necesario re-diseñar todo el producto para que pueda trabajar con Parallel-Editor.

Observaciones a tener en cuenta:

\subsubsection{API’s gráficas y Threads}

La mayoría de las API’s disponibles para hacer interfaces gráficas desktop fueron diseñadas en un 
modo no thread-safe. Es decir, que no se puede garantizar un estado consistente si los objetos que pertencen a la API son
accedidos desde múltiples hilos simultaneamente. Por esta razón, todas las actualizaciones a las ventanas, botones, widgets
en general son realizados por único thread denominado GUI Thread (o dispatching thread). Esta decisión de diseño permite que
un desarrollador no deba preocuparse en la complejidad de manejar algoritmos y estructuras de sincronización
(locks, deadlocks, etc.) que luego son difíciles de testear.

La solución desarrollada en la presenta usa diversos actores (que generalmente corren en un pool de threads) para poder 
responder asincronicamente a mensajes que reciben desde el kernel. Algunos de estos mensajes requieren que se produzcan
cambios en widgets, por ejemplo al recibir un mensaje que agrega texto al sector de edición. Estos cambios deben realizar en
el GUI Thread y no en el Thread que recibe la operación.

Generalemente, las API’s gráficas poseen mecanismos para postergar la ejecución de código para que se ejecute en el Thread de
la GUI y evitar los problemas de sincronización. Esto puede observarse en el siguiente diagrama:

En este diagrama se observa el problema descripto
% [[ diagrama de threads 1]]

En este diagrama se muestra la solución
% [[ diagrama de threads 2]]

\subsubsection{Editor de texto}
Si se desea incorporar la solución en un nuevo editor de texto (por ejemplo un IDE) es necesario tener presente 
las siguientes consiedaraciones:

	\begin{itemize}
		\item Se deberá implementar la interfaz DocumentData que ofrece un mecanismo para agregar y modificar texto en el editor. 
		A su vez ofrece un mecanismo para mantener la posición del cursor consistente y no sea movida por inserciones o
		borrado de texto por otros parcipantes. Si no soporta selección de texto o no existe un cursor, es posible ignorar
		las llamadas a estos métodos.
		La interfaz documentdata se encuentra en el siguiente archivo: ar/noxit/paralleleditor/common/operation/DocumentData.scala.
		\item  Se deberá implementar un listener que atrape los eventos del widget de edición de texto y los propague al kernel.
		La mayoría de las API’s gráficas proveen un mecanismo similar.
	\end{itemize}


\newpage
\begin{thebibliography}{9}
	\bibitem{visiontpprof}
	Ciancio, Gilioli,
	\emph{Visión Trabajo Profesional}.
	Facultad de Ingeniería.
	Universidad de Buenos Aires. 

	\bibitem{jupiter}
	Nichols, Curtis, Dixon and Lamping,
	\emph{High-Latency, Low-Bandwidth Windowing in the Jupiter Collaboration System}.
	Xerox PARC.

	\bibitem{scrum}
	Ken Schwaber,
	\emph{Agile Software Development with Scrum}.
	Prentice Hall, 
	Octubre 2001.
	
	\bibitem{googledocs}
	\emph{Google Docs}. 
	Google Inc., 
	\textsl{http://docs.google.com}.
	
	\bibitem{googlewave}
	\emph{Google Wave}. 
	\textsl{http://wave.google.com}.

	\bibitem{beeweevee}
	Corvalius,
	\emph{BeWeeVee}. 
	\textsl{http://www.beweevee.com}.
	
	\bibitem{cola}
	Mustafa K. Isik,
	\emph{COLA}. 
	\textsl{ http://wiki.eclipse.org/RT\_Shared\_Editing }.
		
\end{thebibliography}

\end{document}